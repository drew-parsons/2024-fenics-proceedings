% --- Auxiliary functions ---

% --- Settinge ---
\newcommand{\bracketed}[1]{\left[#1\right]}

% --- General ---
% Absolute value
\NewDocumentCommand{\abs}{m}{
    % #1: argument
    \left\vert#1\right\vert
}

% --- Linear Algebra ---
% Inner product
\NewDocumentCommand{\iprod}{o m m}{
    % #1: the domain (for integration)
    % #2: argument 1
    % #3: argument 2
    \left(#2,\: #3\right)\IfValueT{#1}{_{#1}}
}

\NewDocumentCommand{\iprodS}{o m m}{
    % #1: the domain (for integration)
    % #2: argument 1
    % #3: argument 2
    \left\langle#2,\: #3\right\rangle\IfValueT{#1}{_{#1}}
}

% Norm
\NewDocumentCommand{\norm}{o m}{
    % #1: Specifier of the norm
    % #2: argument
    \left\vert\left\vert#2\right\vert\right\vert\IfValueT{#1}{_{#1}}
}

% --- Basic tensor functions
% Transposition
\NewDocumentCommand{\transpT}{o m}{
    % #1: Indices of the transposition
    % #2: The tensor
    #2\IfNoValueTF{#1}{^\mathrm{T}}{^{\overset{#1}{\mathrm{T}}}}
}

% % The trace
% \NewDocumentCommand{\tr}{o m}{
%     % #1: True, if brackets should be used
%     % #2: The tensor
%     \mathrm{tr}\IfNoValueTF{#1}{\: #2}{\bracketed{#2}}
% }

% Additive decomposition
\NewDocumentCommand{\sym}{o m}{
    % #1: True, if brackets should be used
    % #2: The tensor
    \mathrm{sym}\IfNoValueTF{#1}{\: #2}{\bracketed{#2}}
}

\NewDocumentCommand{\skw}{o m}{
    % #1: True, if brackets should be used
    % #2: The tensor
    \mathrm{skw}\IfNoValueTF{#1}{\: #2}{\bracketed{#2}}
}

% \NewDocumentCommand{\as}{o m}{
%     % #1: True, if brackets should be used
%     % #2: The tensor
%     \mathrm{as}\IfNoValueTF{#1}{\: #2}{\bracketed{#2}}
% }

% Determinant
\NewDocumentCommand{\determ}{o m}{
    % #1: True, if brackets should be used
    % #2: The tensor
    \mathrm{det}\IfNoValueTF{#1}{\: #2}{\bracketed{#2}}
}

% Inverse
\NewDocumentCommand{\inv}{o m}{
    % #1: True, if brackets should be used
    % #2: The tensor
    \IfNoValueTF{#1}{#2 ^{-1}}{\bracketed{#2}^{-1}}
}

\NewDocumentCommand{\invt}{o m}{
    % #1: True, if brackets should be used
    % #2: The tensor
    \IfNoValueTF{#1}{#2 ^{-\mathrm{T}}}{\bracketed{#2}^{-\mathrm{T}}}
}

% --- Analysis ---
% Gradient and Divergence 
\newcommand{\grad}{\nabla\:}
\renewcommand{\div}{\nabla\cdot}

\NewDocumentCommand{\divc}{o m m}{
    % #1: True, if brackets should be used
    % #2: The arguemnt
    % #3: Configuration, within which the divergence is evaluated
    \nabla_{\vec{#3}} \cdot \IfNoValueTF{#1}{#2}{\bracketed{#2}}
}

\NewDocumentCommand{\gradc}{o m m}{
    % #1: True, if brackets should be used
    % #2: The arguemnt
    % #3: Configuration, within which the divergence is evaluated
    \nabla_{\vec{#3}} \: \IfNoValueTF{#1}{#2}{\bracketed{#2}}
}

% Integrational domains
\newcommand{\dInt}[1]{\;\mathrm{d}#1}
\newcommand{\dVol}{\;\mathrm{d}v}
\newcommand{\dVolR}{\;\mathrm{d}V}
\newcommand{\dSurf}{\;\mathrm{d}a}
\newcommand{\dSurfR}{\;\mathrm{d}A}

% Material timederivative
\NewDocumentCommand{\mtDiff}{o m}{
    % #1: True, if brackets should be used
    % #2: The arguemnt
    \IfNoValueTF{#1}{#2'}{\left(#2\right)'}
}

% Gateaux derivative
\newcommand{\dGat}[2]{\mathrm{D}_{#2}\left[#1\right]}

% Linearisation
\newcommand{\Lin}[1]{\mathrm{LIN}\left[#1\right]}

% --- Function spaces ---
\NewDocumentCommand{\VecSpace}{o o o m}{
    % #1: The domain
    % #2: The boundary condition
    % #3: The dimension
    % #4: The function-space
    \IfNoValueTF{#3}
    {\mathrm{#4} \IfValueT{#2}{_{#2}} \ifthenelse{\equal{#1}{}}{}{\left(#1\right)}}
    {\left( \mathrm{#4} \IfValueT{#2}{_{#2}} \ifthenelse{\equal{#1}{}}{}{\left(#1\right)} \right) ^{#3}}
}

\NewDocumentCommand{\LiiSpace}{o o o}{
    % #1: The domain
    % #2: The boundary condition
    % #3: The dimension
    \IfNoValueTF{#3}
    {\mathrm{L}^2 \IfValueT{#2}{_{#2}} \ifthenelse{\equal{#1}{}}{}{\left(#1\right)}}
    {\left( \mathrm{L}^2 \IfValueT{#2}{_{#2}} \ifthenelse{\equal{#1}{}}{}{\left(#1\right)} \right) ^{#3}}
}

\NewDocumentCommand{\HiSpace}{o o o}{
    % #1: The domain
    % #2: The boundary condition
    % #3: The dimension
    \IfNoValueTF{#3}
    {\mathrm{H}^1 \IfValueT{#2}{_{#2}} \IfValueT{#1}{\left(#1\right)}}
    {\left( \mathrm{H}^1 \IfValueT{#2}{_{#2}}\IfValueT{#1}{\left(#1\right)} \right) ^{#3}}
}

\NewDocumentCommand{\HdivSpace}{o o o}{
    % #1: The domain
    % #2: The boundary condition
    % #3: The dimension
    \IfNoValueTF{#3}
    {\mathrm{H} \IfValueT{#2}{_{#2}} \left(\mathrm{div}\ifthenelse{\equal{#1}{}}{}{,\, #1}\right)}
    {\left( \mathrm{H} \IfValueT{#2}{_{#2}} \left(\mathrm{div}\ifthenelse{\equal{#1}{}}{}{,\, #1}\right) \right) ^{#3}}
}

% --- Finite element spaces ---
\NewDocumentCommand{\sdisc}{m}{
    % #1: The field variable
    #1_{\h}
}

\NewDocumentCommand{\stdisc}{m}{
    % #1: The field variable
    #1_{\h\tau}
}

\NewDocumentCommand{\feP}{o o}{
    % #1: The degree
    % #2: The dimension
    \IfNoValueTF{#2}
    {\mathrm{P} \IfValueT{#1}{_{#1}}}
    {\left( \mathrm{P} \IfValueT{#1}{_{#1}} \right)^{#2}}
}

\NewDocumentCommand{\feDP}{o o}{
    % #1: The degree
    % #2: The dimension
    \IfNoValueTF{#2}
    {\mathrm{DP} \IfValueT{#1}{_{#1}}}
    {\left( \mathrm{DP} \IfValueT{#1}{_{#1}} \right)^{#2}}
}

\NewDocumentCommand{\feRT}{o o}{
    % #1: The degree
    % #2: The dimension
    \IfNoValueTF{#2}
    {\mathrm{RT} \IfValueT{#1}{_{#1}}}
    {\left( \mathrm{RT} \IfValueT{#1}{_{#1}} \right)^{#2}}
}

\NewDocumentCommand{\feBDM}{o o}{
    % #1: The degree
    % #2: The dimension
    \IfNoValueTF{#2}
    {\mathrm{BDM} \IfValueT{#1}{_{#1}}}
    {\left( \mathrm{BDM} \IfValueT{#1}{_{#1}} \right)^{#2}}
}